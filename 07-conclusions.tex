\section{Conclusions}
In this work we have presented a class of metrics for cache studies
that provide deeper insight into the behavior of the replacement
algorithm when used for a particular application.
When Belady's algorithm is used, one can discover inflection points
that cause poor cache performance in the application.
A comparison between TTR for Belady's algorithm and any set of
replacement policies can provide insight for cache designers to choose
replacement policies that are appropriate for the class of
applications that the designer deems important.
Furthermore, our sampling technique allows for quick iteration in
policy design with rapid feedback to the developer.
MATR was reasonably straightforward to capture in our simulations, but
other types of TTR may be more useful or easy to record in other
contexts.

TTR differs from traditional metrics like IPC and MPKI by providing a qualitative resource for cache performance.
IPC and MPKI only indicate how well a replacement policy performs, but not where it performs badly.
In contrast, TTR provides replacement policy designers insight on how to improve their policy.

While we have presented these metrics in the context of CPU caches,
the potential application for these techniques extends beyond CPU
cache design.
For example, large web applications employ DRAM as a cache for large
databases held on arrays of disc drives as a method for increasing
input and output operation throughput.
Additionally, other application specific integrated circuits and
processors that utilize caching may find TTR useful.
In order to assist others in using our techniques, we have released
our source code publicly (reference withheld) and encourage revisions
and additions as appropriate for other domains.
